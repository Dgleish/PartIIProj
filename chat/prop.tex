\documentclass[a4paper, 12pt]{article}
\usepackage[a4paper, margin=3cm]{geometry}
\usepackage{graphicx}
\usepackage{enumitem}
\usepackage{parskip} % Use newline to separate paragraphs

 % Use symbols for footnotes to disambiguate from citations and reset counter
 % on each page
\usepackage[perpage,symbol]{footmisc}
\usepackage[utf8]{inputenc} % Expand unicode characters into latex commands

\renewcommand{\thefootnote}{\fnsymbol{footnote}}

\begin{document}
  \begin{titlepage}

    \begin{minipage}[t][][t]{0.5\textwidth}
      %\includegraphics[width=40mm]{./figures/uclogo.pdf}
    \end{minipage}
    \begin{minipage}{0.5\textwidth}
      \begin{flushright}
        \large
        \textit{Daniel Chatfield}
        \\
        \textit{Robinson College}
        \\
        \texttt{\textit{dc584}}
      \end{flushright}
    \end{minipage}

    \vfill

    \begin{center}

      {\scshape\Large Part II Project Proposal}
      \vspace{1.5cm}

      {\huge\bfseries Elliptic Curve Digital Signature Protection for Mifare
      Classic RFID Cards\par}
      \vspace{1cm}

      {\large \today}

      \vfill
  	  supervised by\par
  	  Dr.~Markus Kuhn

    \end{center}

    %{\bf Project Supervisor:} Dr.~Markus Kuhn
    %\vspace{0.2cm}

    %{\bf Director of Studies:} Alan Mycroft
    %\vspace{0.2cm}

    %{\bf Project Overseers:} Marcelo Fiore \& Ian Leslie
  \end{titlepage}

  %\addtocounter{page}{1}

  \section*{Introduction}

  Many organizations use contactless smart cards for authentication, including
  the University of Cambridge. The Mifare Classic card is one of the most
  popular smart cards and is currently the card used by the University of
  Cambridge for everything from access to buildings to paying for meals.

  The communication between the card and the reader is based on the ISO-14443-A
  standard with a proprietary authentication scheme that provides mutual
  authentication between the reader and the card. The proprietary authentication
  scheme has been the subject of academic research for several years. Initial
  research focussed on reverse engineering the protocol and then later research
  exploited weaknesses to fully compromise the cards~\cite{tan2009practical}.

  The storage on the card is split into sectors, each sector has two keys
  associated with it (A and B) each of which can be configured to allow
  read/write access permissions. A typical deployment (like Cambridge University
  cards) uses one key as a read/write key that is kept secret by the card
  issuing office and the other key for reading that is distributed to the
  various organzations that need to read the cards.

  Today it is possible to extract all the secret keys from a Mifare Classic card
  within 10 minutes if just one of the sectors uses a default
  key\footnotemark{}. Since it is unusual for all the sectors to have unique
  keys it is usually much quicker. Once the keys have been extracted, the
  attacker can read/write to the card and any other cards that have the same
  keys.

  \footnotetext{The Mifare specification requires this for the first sector as it should contain data for identifying who issued the card.}

  The publication of these vulnerabilities prompted some organizations to move
  to more secure contactless smart cards but many are still using Mifare Classic
  cards due to the cost or difficulty involved in replacing all issued cards or
  upgrading readers.

  I plan to improve the security of these cards by digitally signing the data
  along with the card UID\@. The reader would then verify that the signature is
  valid for the data on the card, thus preventing unauthorized modification. By
  including the card UID in the payload to be signed it substantially raises the
  bar for cloning a card as it would require the possession of either a Mifare
  Classic compatible card that allowed the UID to be set or a device that could
  emulate one such as a Proxmark 3. Elliptic curve digital signatures are
  appropriate as they offer much shorter signatures than traditional digital
  signature algorithms, this is desirable with the memory constraints of the
  card.

  It is common to have the need to disable a card before it expires (for
  example, when an employee leaves or when a card is lost). Some door-access
  readers are ``online''\footnote{They communicate with some central server to
  establish whether they should grant access.} and thus all that is required to
  revoke access to these is to remove the card from the access list, however a
  large proportion are ``offline''\footnotemark{} and require a manual update to
  the reader to block access to a specific card. I will design and implement a
  protocol for using the cards themselves as a medium for propagating the list
  of revoked cards from online readers to offline readers.



  \footnotetext{They either have a preprogrammed access list or the card has a
  signed token.}

  \section*{Starting Point}

  I will draw upon the following during my project:

  \begin{itemize}
    \item \textbf{Computer Science Tripos} \newline
      My project will use knowledge from parts 1A and 1B of the tripos. The
      digital signature implementation will draw from content from the Security
      I course. Theory from the 1B Networking course will be useful when
      designing the offline revocation list distribution. The part II Computer
      Systems Modeling course may prove useful for modeling the efficiency and
      reliability of my distribution protocol in different environments.

    \item \textbf{Experience} \newline
      I have some experience with Mifare Classic cards and NFC more broadly and
      extensive experience with the Go programming language.

    \item \textbf{Go crypto libraries} \newline
      Go has comprehensive cryptography libraries that implement much of the
      required elliptic curve cryptography required for this project.
  \end{itemize}

  \section*{Project Structure}

  \subsection*{Core library in Go}

  As well as supporting authenticating to, reading from and writing to Mifare
  Classic cards, the core library will support the writing and verification of
  elliptic curve digital signatures allowing a reader to perform offline
  verification of a card without holding the private key used to sign it.

  I have chosen Go as the language to implement the core library as it compiles
  to statically linked binaries that don't require a runtime, has excellent
  support for cryptographic operations and its channels are a natural fit for
  the asynchronous nature of NFC communication.

  Go can be easily cross compiled for different platforms including Android and
  iOS, making the codebase very portable.

  The development of the library will follow software engineering best
  practices. I will develop a comprehensive test suite comprising of unit tests
  to test individual methods, and acceptance tests that test high-level
  functions like writing an entire card and then reading it. This will help
  ensure correctness and reduce the likelihood of regressions being introduced
  as features are added.

  The library will provide abstractions to make it easy to perform common tasks
  in third party projects, without being familiar with how NFC works.


  \subsection*{Command Line Interface}

  The library will be wrapped in a command line interface that exposes its
  high-level functionality. The CLI will have two modes --- one suitable for
  consumption by a human using a terminal, and one that returns machine readable
  JSON to make it easy for projects not written in Go to use the library without
  attempting to both parse the text output and rely on it being backwards
  compatible in future versions.


  \subsection*{Desktop Application}

  A desktop application will sit on top of the CLI providing a graphical
  interface for managing the lifecycle of the cards including the writing,
  reading and verification. This desktop application will be developed using
  Github's Electron~\cite{electron}, as this makes it easy to make
  cross-platform applications with rich UIs.


  \subsection*{Offline Revocation List Propagation Protocol}

  The development of the offline propagation protocol will be iterative. I will
  first make a simple and naïve protocol and create a simulation suite that
  benchmarks this protocol in different environments\footnotemark{}. I will then
  evolve the design to overcome weaknesses, using the simulation suite results
  to justify and drive the changes made.

  \footnotetext{I will vary the number of cards and readers (including
  offline/online split). I will also simulate different card access patterns in
  an attempt to both simulate likely real world usage and also identify
  pathological scenarios.}

  \section*{Possible Extensions}

  I'm sure that further possible extensions will become apparent whilst carrying
  out the project but two possibilities are:


  \subsection*{Stronger protection of the private key}

  The security of the system relies on the private key remaining private. To
  help ensure this I could extend the application to support the use of a
  hardware security module that would prevent the key from being covertly
  extracted from one of the machines that sign the cards.

  Alternatively, I could allow the application to connect to a central server to
  do the signing. This would allow an organization to distribute the application
  to trusted employees without entrusting them with the key as they would have
  to sign in to the central server\footnotemark{} which could then log what was
  signed by which employee and access could be revoked immediately.

  \footnotetext{Cambridge University could use raven for this purpose}


  \subsection*{Offline access violation reporting}

  As I have already stated, many readers are offline and cannot report access
  violations. Typically the only way to discover such violations is for a
  technician to physically download the access log\footnote{Not all readers
  store a log} from the reader. I could extend my project to allow these doors
  to use the cards as a medium to report these violations back to an online
  reader.


  \section*{Success Criteria}

  At the end of the project I should be able to:
  \begin{itemize}
    \item Generate new public/private keypairs for use with a new card
      deployment (or upgrading an existing deployment to support signatures).
    \item Load an existing private key or public key.
    \item Provision a new card with a digital signature.
    \item Upgrade an existing card to include a digital signature.
    \item Read a card with a digital signature and verify that the signature
      matches.
    \item Reject cloned cards where the UID hasn't been spoofed.
    \item Revoke a card and use the cards to propagate this to offline readers.
  \end{itemize}

  \section*{Project Timetable}

  \subsection*{Michaelmas term}
  \begin{description}
    \item[24th Oct --- 8th Nov] \hfill
      \begin{enumerate}
        \item Obtain an NFC reader.
        \item Obtain some blank Mifare Classic cards.
        \item Research elliptic curve cryptography.
        \item Familiarize myself with libnfc.
      \end{enumerate}

    \item[9th Nov --- 22nd Nov] \hfill
      \begin{enumerate}
        \item Research published weaknesses in Mifare Classic cards.
        \item Develop a threat model against the existing system to use as a
          benchmark for comparison.
        \item Begin development of the core library.
      \end{enumerate}

      \textbf{Milestone 1}
      By 22nd November I should be able to read and write to a Mifare card using
      my library.

    \item[23rd Nov --- 6th Dec] \hfill
      \begin{enumerate}
        \item Develop the command line wrapper around the library.
        \item Test the signature creation and verification using Mifare cards.
      \end{enumerate}
  \end{description}

  \subsection*{Christmas vacation}
  \begin{description}
    \item[6th Dec --- 10th Jan] \hfill \\
      Start development of the desktop application.
  \end{description}

  \subsection*{Lent term}
  \begin{description}
    \item[11th Jan --- 17th Jan] \hfill \\
      Write progress report and complete desktop application.


      \textbf{Milestone 2}
      By 17th January I should be able to use the desktop application to read
      and write to the cards and verify the digital signatures.

    \item[18th Jan --- 31st Jan] \hfill
      \begin{enumerate}
        \item Design a protocol for offline propagation of the card revocation
          list.
        \item Use modeling to verify that the design works at scale.
      \end{enumerate}

    \item[1st Feb --- 14th Feb] \hfill \\
      Implement offline revocation list propagation.

    \item[15th Feb --- 13th Mar] \hfill \\
      Main chapters of dissertation

  \end{description}

  \subsection*{Easter Vacation}
  \begin{description}
    \item[14th Mar --- 17th Apr] \hfill \\
      Proof read dissertation.

      \textbf{Milestone 3}
      By the end of the Easter vacation I should have delivered the full draft
      proposal to my supervisor and DoS.
  \end{description}


  \section*{Resources Required}

  To complete the project I will need an ISO-14443-A compatible USB NFC reader
  and some blank Mifare Classic cards. I don't envisage any problems sourcing
  these as they are both widely available at relatively low cost.

  Development and testing will be done on my 2013 MacBook Pro, I accept full
  responsibility for this machine and I have made contingency plans to protect
  myself against hardware and/or software failure.

  \section*{Backup Policy}

  I will use git for all source code (including latex), using either github or
  bitbucket as a remote. In addition to this, I will use Apple time machine to
  periodically backup all files on my laptop. In the event of hardware failure,
  I will only lose any changes made since the most recent git push or time
  machine backup which should be no more than a days work.

  Any computer that supports compiling Go is suitable for development and
  therefore should my laptop become permanently broken it will be easy to source
  a replacement.

  \bibliographystyle{ieeetr}
  \bibliography{bibliography}

\end{document}
